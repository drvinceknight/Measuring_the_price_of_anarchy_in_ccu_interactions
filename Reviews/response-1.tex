\documentclass{article}

\usepackage{fullpage}
\usepackage{parskip}
\usepackage{booktabs}
\usepackage{minted}
\usepackage{amsmath}

\newcommand{\NH}{\text{NH}}
\newcommand{\RG}{\text{RG}}

\title{Responses to second review for Measuring the Price of Anarchy in Critical Care Unit Interactions}
\author{Vincent Knight*
    \and
        Izabela Komenda
    \and
        Jeff Griffiths}

\begin{document}

\maketitle

We are grateful for positive comments and address each in turn below.

\subsection{Reviewer 1}

\begin{quote}
    \begin{textit}{
            The current version of the paper is accepted for publication as all
            the quarries are justified properly in the revised paper. Thanks a
            lot.
    }\end{textit}
\end{quote}

No action taken. We are grateful to the reviewer for their time and expertise.

\subsection{Reviewer 2}

\begin{quote}
    \begin{textit}{
It becomes more confusing especially to note the relation between H and h. Why not define KH, h, H ,l etc in this table? This will be easy for the reader to refer to this table easily rather than having to move up and down to get some definitions. When I got to the end of the document, the discussions, I had already forgotten some definitions of some unknowns and I had to spent almost a minute looking for their definitions in the document. That is my point. I suggest that the authors define all the unknowns in a single table. It makes the paper more readable than its current form.
    }\end{textit}
\end{quote}

As highlighted in our previous response we do not feel that this is a request
that would improve the paper. The general parameters for the problem itself are
listed, the description of the solution approach then defines all relevant
parameters (this is not unusual).

\subsection{Reviewer 3}

\begin{quote}
    \begin{textit}{
1. p. 10, l. 2: "Recalling equations (4-5) and Figure 9 have:"
Not grammatical. "have" -> "gives" ?
    }\end{textit}
\end{quote}

This has been changed.

\begin{quote}
    \begin{textit}{
        2. p. 10, second last par.: "Note that intuitively..."
        Are the authors able to concisely provide some intuition as to why
        the naive result is not correct?
    }\end{textit}
\end{quote}

We expect that this is due to the hospitals running at full capacity which
implies more overall rejections (the queues are loss queues). We are however
hesitant to include that in the paper itself. We feel that to do so, a full
investigation should be carried out (which we feel is outside of the scope of
this paper).

\begin{quote}
    \begin{textit}{
        3. Just before 3.1: "Kernel"
        Why is kernel capitalized?
    }\end{textit}
\end{quote}

This has been changed.

\begin{quote}
    \begin{textit}{
        4. p. 11, sentence just before equation: This models corresponds to
        the first..."
        Not grammatically correct.
    }\end{textit}
\end{quote}

This has been modified to read:

``This models corresponds to the first
possibility: the patient is lost (from the point of view of this model).''


\begin{quote}
    \begin{textit}{
        5. p. 16, second bullet:
        Believe that "Multi player" is one word.
    }\end{textit}
\end{quote}

This has been fixed.

\subsection{Editor requests}

\begin{quote}
    \begin{textit}{
Referee \#2 finds the notation difficult to follow and suggests the addition of a list of parameters. I think the difficulty is perhaps exaggerated as I find the text that the Referee quotes from p3 relatively easy to follow. However, I would ask the authors to consider whether a list of parameters and key notation would be of benefit to the reader.
    }\end{textit}
\end{quote}

We have considered it and do not feel it would help very much. The problem
parameters are listed in the table, everything else is part of the problem
solution which we feel is clearly described.

\begin{quote}
    \begin{textit}{
        The journal uses a Harvard-style convention for references. Please can the authors modify the citation of references to reflect this.
    }\end{textit}
\end{quote}

This has been changed throughout.

\begin{quote}
    \begin{textit}{
        More informative Figure captions might help with the notation issue. For
        example, "Figure 7: The effect of \(K_{RG}\) on \(\overline{\lambda}\)" would be easier
        to understand if "\(K_{RG}\)" and "\(\overline{\lambda}\)" were explained in words.
    }\end{textit}
\end{quote}

The captions have been changed to be more descriptive.

\begin{quote}
    \begin{textit}{
        The size of text used in images is a potential issue. For example, I am not
        sure that the detailed annotations in Figure 3 will be legible when the figure
        is prepared for inclusion in the journal. Could the two graphs in Figure 7 be
        replaced by a single graph? The only difference between the two graphs seems to
        be a minor change in annotation. Similarly for Figure 8. The authors should try
        to provide images of the highest quality possible as separate files.
    }\end{textit}
\end{quote}

Figure 3 has been made slightly larger but does indeed have quite small print
but in electronic format can be zoomed in on to any level (see later comment
about vector format of images).

We have removed the second plot for Figures 7 and 8 and added an explanation in the caption saying that the same effect holds for the other values of \(\lambda\).

Image file formats are in vector format (converted from svg to pdf) which
ensures lossless quality, this will all be provided as separate files.

\end{document}
