\documentclass{article}

\usepackage{fullpage}
\usepackage{parskip}

\title{Responses to Reviews for Measuring the Price of Anarchy in Critical Care Unit Interactions}
\author{Vincent Knight*
    \and
        Izabela Komenda
    \and
        Jeff Griffiths}

\begin{document}

\maketitle

Below we respond to each reviewer in detail.

\subsection{Reviewer 1}

\begin{quote}
    \begin{textit}{
Write the clarifications of notations \(\lambda_{h}^{(l,h)},\lambda_{h}^{(l,l)}\). One \(h \in (NH,RG)\) other \(h\) denotes RG experiences high demand.
    }\end{textit}
\end{quote}

\begin{itemize}
    \item Change one of the hs.
\end{itemize}

\begin{quote}
    \begin{textit}{
Give explanation for figure 3.
    }\end{textit}
\end{quote}

\begin{itemize}
    \item Add a sentence.
\end{itemize}

\begin{quote}
    \begin{textit}{
    Write clearly major contribution of the paper corresponding to the existing
    literature.
    }\end{textit}
\end{quote}

\begin{itemize}
    \item Point out that this is there. Ask if there would be a better way of doing it?
\end{itemize}

\begin{quote}
    \begin{textit}{
    State clearly the objectives of the model.
    }\end{textit}
\end{quote}


\begin{itemize}
    \item Not sure I understand, mention that have clarified the model for each player.
\end{itemize}

\subsection{Reviewer 2}

\begin{quote}
    \begin{textit}{
    On page 5, the model is referred to as a Figure 6. This confuses and leaves a question on whether it is a figure or equations. I am unsure whether figure 6 is missing or authors are referring to the model on page 6.
    }\end{textit}
\end{quote}

\begin{itemize}
    \item Mention that this has changed.
\end{itemize}

\begin{quote}
    \begin{textit}{
The paper is unreadable. The unknowns used, some, are undefined, for example on page 3, h, ch, etc are not defined and it is very difficult to find definitions of those defined. I suggest that the authors provide a list of the parameters used somewhere in the paper where it is easy to refer.
    }\end{textit}
\end{quote}

\begin{itemize}
    \item Highlight that this is all in the paper.
\end{itemize}

\begin{quote}
    \begin{textit}{
    On page 12, authors states that ``It is also noted that as demand increases the effect of uncoordinated behaviour increases (and the recommended target also increases) as shown in Figure 13'', is there any possible explanation to this finding.
    }\end{textit}
\end{quote}

\begin{itemize}
    \item Suggest some explanations.
\end{itemize}

\begin{quote}
    \begin{textit}{
    Are the Queuing and Game theoretic models never been implemented to solve such problems? There is a need to review literature that has information on application of the used models to hospital problems or related problems.
    }\end{textit}
\end{quote}

\begin{itemize}
    \item This is in the paper.
\end{itemize}

\begin{quote}
    \begin{textit}{
The authors need to give a justification on the reason why they are implementing the models.
    }\end{textit}
\end{quote}

\begin{itemize}
    \item Point to wider discussion invited by reviewer 3.
\end{itemize}

\subsection{Reviewer 3}


\begin{quote}
    \begin{textit}
        {I recommend that the authors write a more complete and coherent
        Discussion and/or Conclusion which places their work in the context of
        the critcal care system.  It should relate model limitations to the
        critical care system. The current way that limitations are discussed
        is too technical. A brief discussion of stakeholder feedback would
        also be useful.}
    \end{textit}
\end{quote}

\begin{itemize}
    \item Add to discussion, point out that previous limitations were technical and now discuss ccu limitations. Mainly following from points suggested by reviewer 3.
\end{itemize}

\begin{quote}
    \begin{textit}
        {
        At the beginning of Section 3.1, the authors claim that if neither CCU
        is able to admit patients, then admission to the CCU is cancelled and
        the patient is admitted to a general ward. In my experience modelling
        CCUs, it is highly unlikely that a critical patient would ever be
        admitted to a general ward.  CCU beds are almost always equipped with
        ventilators and ward beds never. I believe that more likely courses of
        action are:
        \begin{itemize}
         \item Sending the patient to another hospital's CCU.  This would be a CCU
           outside of the model.
         \item ``Bumping" a patient currently in the CCU to a ward bed, to free up
           a bed. This would normally be a patient who was nearly ready to be
           transferred to the ward anyway.
         \item Accommodating the patient in the hospital's post-anesthesia care
           unit (PACU), which sometimes functions as an overflow for the CCU.
        \end{itemize}
        I recommend that the authors consult with their stakeholders and
        correct this comment in their paper, if necessary.  This wouldn't have
        any impact on their model, but it is important to use the correct
        language.
        }
    \end{textit}
\end{quote}

\begin{itemize}
    \item List all of above as options but use first as most related to model.
\end{itemize}

\begin{quote}
    \begin{textit}
        {
        The axis labels on many of the graphs are in a very small font. The
        font size should be increased.
        }
    \end{textit}
\end{quote}

All axis labels have now got a bigger font. We appreciate this comment: if any
other aspects of the graphics could be improved we would be happy to make the
suggested changes.

\begin{quote}
    \begin{textit}
        {
        The optimisation problem is shown as a figure (Figure 6). I suggest
        that this be separated from the text with a header such as
        ``Optimisation Problem" (much like a theorem).  Calling it a figure
        seems like a misnomer, and for me, actually made it harder to find.
        }
    \end{textit}
\end{quote}

\begin{itemize}
    \item Change this.
\end{itemize}

\begin{quote}
    \begin{textit}
        {
        In Figure 7, the caption should state which points are \(f_NH\) and
        which are \(F_RG\).
        }
    \end{textit}
\end{quote}

\begin{itemize}
    \item State this in captions.
\end{itemize}

\begin{quote}
    \begin{textit}
        {
I was puzzled as to whether the target, t, appeared in the definition
of \(T^*\). As I understand it, \(T^*\) is independent of \(t\). Is this true?
Also, for which values of \(K_{NH}\) and \(K_{RG}\) is the maximum achieved? Is my
intuition correct that it is achieved at \(K_{NH} = c_{NH}\) and \(K_{RG} = c_{RG}\)?
        }
    \end{textit}
\end{quote}

\begin{itemize}
    \item Clarify this, I believe that it's not always the case that \(K_{NH}=c_{NH}\).
\end{itemize}

\begin{quote}
    \begin{textit}
        {
The authors make extensive use of bullets in the text. While bullets
can be useful to draw attention to key points, they can lead to lazy
writing. In my view, converting some of the bullet lists would improve
the clarity of the exposition.
        }
    \end{textit}
\end{quote}

\begin{itemize}
    \item Remove bullet points.
\end{itemize}
\end{document}
